% !TEX root = ./iclr2026_conference.tex
\newpage

\section{Proof of Ergodic Properties and Limit Theorems\\
(Theorem~\ref{thm: efficient_stats}) \& (Theorem~\ref{thm: LLN_CLT_res})
}
%\section{Proof of Theorem~\ref{thm: efficient_stats}}
\label{app: efficient_stats}
We start by proving a series of properties that the induced Markov Chain satisfies, that will be necessary for proving the Theorem~\ref{thm: efficient_stats}. 
%%%%%%%%%%%%%\subsection{Small \& Petite Sets}
\begin{proposition} [Proposition 5.5.3 \citep{meyn2012markov}]\label{prop: petite_set}
    If a set $C \in \mathcal{B}(\R^d)$ is $\nu_m$-small, then it is $\nu_{\delta_m}$-petite for some $\delta_m > 0$.
\end{proposition}
\noindent\textit{Intuition.}  
The notions of \emph{small} and \emph{petite} sets are technical tools in Markov chain theory that help verify stability properties.  
A set $C$ is called \emph{small} if, starting from $C$, the chain has a uniform positive chance of reaching any region of the state space within a fixed number of steps.  
A \emph{petite} set is a weaker concept: instead of requiring such uniformity in a single time step, it allows the chance of hitting any region to be distributed over a random number of steps (via a probability distribution over times).  
Thus, every small set is automatically petite, but the reverse is not true.  
Intuitively, small sets guarantee “uniform mixing after a fixed horizon,” while petite sets guarantee the same effect “on average over time.”  

% \begin{lemma}
% \label{prop: bound prr epoch deviations}
% Let Assumption~\ref{assumpt: lipschitz} hold. For any fixed $d \in [n]$ and $x \in \R^{d}$, the following inequality holds: 
% \begin{eqnarray}
% \mathbb{E}_{\pi \in \mathcal{S}} \left[\norm{\frac{1}{d} \sum_{j=0}^{d-1} F_{\omega_j}(x)- F(x)}^4 \right] &\le& 
% \frac{1}{k^4}\left[
% \frac{k}{n}\,S_4
% +\frac{9k(k-1)}{n(n-1)}\Big(n^2(\operatorname{tr}\hat\Sigma)^2 - S_4\Big)
% \right].
% \end{eqnarray} 
% \vspace{-.2cm}
%  where $\mathcal{S}$ is the set of all permutations of the set $[n]$ of length $d$, 
%  $A =\frac{2}{n} \sum_{i=1}^{n} L_i^2$ and $\sigma_*^2 = \frac{1}{n} \sum\limits_{i=1}^{n} \|F_i(z^*)\|^2$.
% \end{lemma}
% \begin{proof}
% First, we substitute in Proposition \ref{prop:random_reshuffling} $X_i \leftarrow F_i(z_0^k), i\in[n]$ and fix an integer $d\in [n]$. Next, we draw a permutation with $d$ elements uniformly at random from the set of all permutations of $[n]$ with $d$ elements, i.e. $\pi \sim \mathcal{U}(\mathcal{S})$. In other words, let $X_{\pi_0}, \dotsc X_{\pi_{d-1}}$ be sampled uniformly without replacement from $\{X_1,\dotsc, X_{n}\}$. Then, we have that the quantities $\Bar{X}, \mu$ from Proposition \ref{prop:random_reshuffling} are equal to: 
% \begin{eqnarray}
%   \Bar{X} = \frac{1}{d}\sum_{j=0}^{d-1} F_{\pi_j}(z), \quad \mu = \frac{1}{n} \sum\limits_{j=1}^{n}F_j(z) =F(z) 
% \end{eqnarray}
% From Proposition \ref{prop:random_reshuffling}, thus, we get that:
% \begin{eqnarray}
% \mathbb{E}_{\pi \in \mathcal{S}} \left[\norm{\Bar{X} - \mu}^2\right] = \mathbb{E}_{\pi \in \mathcal{S}} \left[\norm{\frac{1}{d}\sum_{j=0}^{d-1} F_{\pi_j}(z) - F(z)}^2 \right]= \frac{n-d}{d(n-1)}\frac{1}{n} \sum_{j=1}^{n}\left\|F_j(z)-F(z)\right\|^2 \label{in_to_substitute_var}
% \end{eqnarray}
% Using Proposition \ref{app: bound for const assumpt}, we next bound the sum on the right hand-sight (RHS) as follows:
% \begin{eqnarray}
% d^2 \mathbb{E}_{\pi \in \mathcal{S}} \left[\norm{\frac{1}{d}\sum_{j=0}^{d-1} F_{\pi_j}(z) - F(z)}^2\right] &\leq& \frac{d(n-d)}{n-1} \left(A \|z - z^*\|^2 + 2\sigma_*^2 \right) \nonumber
% \end{eqnarray}
%  \end{proof}
%%%%%%%%%%%%%
\subsection{Proof of Continuous State Time Homogenious Markov Chain\\(Lemma~\ref{lem:epoch-homog})}
\begin{lemma}[Epoch-level homogeneity and kernel]
Fix $\gamma>0$ and $n\in\mathbb{N}$. Then Perturbed-SGD can be described at each epoch $k$ as:
\emph{Draw $\omega_k$ uniformly from $\mathfrak{S}_n$ and set}
\[\vspace{-0.25em}
x_{k+1}
\;=\;
H(x_k,\omega_k)\;+\;U_k, 
\quad U_k\sim \mathcal N(0,\Sigma),
\vspace{-0.25em}\]
where $H(x,\omega)$ denotes the endpoint of one reshuffled pass started at $x$ with permutation $\omega$ (i.e., the map induced by $n$ inner updates with step size $\gamma$). Then $(x_k)_{k\ge 0}$ is a time-homogeneous Markov chain on $\R^d$ with transition kernel
\[
P(x,A)
\;=\;
\frac{1}{n!}\sum_{\omega\in\mathfrak{S}_n}
\int_A \phi\!\big(y;\,H(x,\omega),\,\Sigma \big)\,dy,
\qquad A\in\mathcal{B}(\R^d),
\]
where $\phi(\cdot;m,\Sigma )$ is the $d$-variate Gaussian density with mean $m$ and covariance $\Sigma$.
\end{lemma}
\begin{proof}
Fix $\gamma>0$ and $n\in\mathbb{N}$. For any $x\in\R^d$ and $\omega\in\mathfrak{S}_n$, define the inner-epoch recursion
\[
x^{[0]}(x,\omega)=x,\qquad
x^{[j+1]}(x,\omega)=x^{[j]}(x,\omega)-\gamma\,F_{\omega[j]}\big(x^{[j]}(x,\omega)\big),\;\; j=0,\ldots,n-1,
\]
and the (measurable) epoch map
\[
H(x,\omega):=x-\gamma\sum_{j=0}^{n-1}F_{\omega[j]}\big(x^{[j]}(x,\omega)\big).
\]
By construction, at epoch $k$ the algorithm updates as
\[
x_{k+1}=H(x_k,\omega_k)+U_k,
\]
where $(\omega_k)_{k\ge0}$ are i.i.d.\ uniform on $\mathfrak{S}_n$ and $(U_k)_{k\ge0}$ are i.i.d.\ with law $\mathcal N(0,\Sigma I_d)$, independent of $(\omega_k)_{k\ge0}$ and of $x_k$ given the present state.

\smallskip
\noindent\emph{Markov property.} Let $A\in\mathcal{B}(\R^d)$. Using the tower property and the independence of $\omega_k,U_k$ from the past given $x_k$,
\[
\Pr(x_{k+1}\in A\mid x_0,\ldots,x_k)
=\E\!\left[\Pr\!\left(H(x_k,\omega_k)+U_k\in A\;\middle|\;x_k,\omega_k\right)\Bigm|x_k\right]
=\E\!\left[\Pr\!\left(H(x_k,\omega)+U\in A\right)\right],
\]
where the outer expectation is over $\omega\sim\mathrm{Unif}(\mathfrak{S}_n)$ and $U\sim\mathcal N(0,\Sigma I_d)$, independent. Thus
\[
\Pr(x_{k+1}\in A\mid x_0,\ldots,x_k)=\Pr(x_{k+1}\in A\mid x_k)=:P(x_k,A),
\]
so $(x_k)_{k\ge0}$ is a Markov chain.

\smallskip
\noindent\emph{Time-homogeneity and kernel.} Since the joint law of $(\omega_k,U_k)$ does not depend on $k$, the transition kernel $P$ is time-invariant. By conditioning on $\omega$ and integrating over the Gaussian $U$,
\[
P(x,A)=\frac{1}{n!}\sum_{\omega\in\mathfrak{S}_n}\Pr\!\left(H(x,\omega)+U\in A\right)
=\frac{1}{n!}\sum_{\omega\in\mathfrak{S}_n}\int_A \phi\!\left(y;\,H(x,\omega),\,\Sigma I_d\right)\,dy,
\]
where $\phi(\cdot;m,\Sigma I_d)$ denotes the $d$-variate Gaussian density with mean $m$ and covariance $\Sigma I_d$. This yields the stated expression for $P$ and establishes time-homogeneity on $\R^d$.

\smallskip
\noindent\emph{Augmented formulation (for reference).} On the product space $\R^d\times\mathfrak{S}_n$, define the next permutation $\omega' \sim \mathrm{Unif}(\mathfrak{S}_n)$ independently of $(x,\omega)$ and $U$. Then the augmented chain $((x_k,\omega_k))_{k\ge0}$ satisfies
\[
(x',\omega')=\big(H(x,\omega)+U,\;\omega'\big),
\]
and the associated kernel is
\[
K\big((x,\omega),A\times B\big)=\int_A \phi\!\left(y;\,H(x,\omega),\,\Sigma I_d\right)dy\cdot \frac{|B|}{n!},
\]
which is manifestly time-homogeneous. \end{proof}

We, next, show that there exists an energy function that describes the iterates of the Markov chain. 
%%%%%%%%%%%%%
\subsection{Proof of Foster-Lyuapunov inequality }\label{app: foster}
A central tool for proving stability and ergodicity of Markov chains is the \emph{Foster–Lyapunov inequality}.  
The idea is to construct an ``energy'' or ``Lyapunov'' function $\mathcal{E}(x,x^*)$ that tracks the distance of the chain’s state from equilibrium.  
If this function decreases on average outside a bounded region, it ensures that the process cannot drift to infinity and will instead return frequently to a compact set.  
This property, when combined with the minorization condition, implies positive Harris recurrence and geometric ergodicity \citep{meyn2012markov}.  

In our case, a natural candidate for such an energy is the squared distance to a solution $x^*$, up to an additive constant.  
The following corollary verifies that this choice indeed satisfies a Foster–Lyapunov inequality for Perturbed SGD–\RRresh,  
showing that the expected energy after one epoch contracts linearly up to a fixed additive term.
\begin{corollary}\label{corol: energy_decrease}
    Let Assumptions~\ref{assumt: sol_set_non_empty}-\ref{assumpt: lipschitz} hold. The function $\mathcal{E}(x_0^{k}, x^*) = \|x_0^{k} - x^*\|_2^2 + 1$ satisfies for any $x^*\in \mathcal{X}^*$ the inequality
    \begin{eqnarray}
        \exof{\mathcal{E}(x^{k+1}_0, x^*)\given \filter_k} &\leq& c_1 \mathcal{E}(x^k_0, x^*) + c_2, \nonumber
    \end{eqnarray}
    where $c_1 = 1 - \frac{\gamma n \mu}{2}$ and $c_2 = \frac{\gamma n \mu}{2} + \frac{8 n \gamma^2 L_{max}^2}{\mu^2} \sigma_*^2 + \frac{8\lambda}{\mu}$.
\end{corollary}
\begin{proof}
    From inequality \eqref{ineq_cond_res} of Theorem~\ref{thm: convergence_rate}, we have that
    \begin{eqnarray}
        \exof{\|x_{k+1}^{0} - x^*\|^2\given \filter_k} &\leq& \left(1 - \frac{\gamma  n\mu}{2}\right)^{k+1} \|x^0_0 - x^*\|^2 + \frac{8 n \gamma^2 L_{max}^2}{\mu^2} \sigma_*^2 + \frac{8\lambda}{\mu} \nonumber\nonumber
    \end{eqnarray}
    Adding in both sides one and using the definition of $\mathcal{E}(x^k_0, x^*),$ we obtain
    \begin{eqnarray}
      \exof{\|x_{k+1}^{0} - x^*\|^2 + 1\given \filter_k} &\leq& \left(1 - \frac{\gamma n \mu}{2}\right) \left(\|x_k - x^*\|^2+1\right) + \frac{\gamma n \mu}{2} + \frac{8 n \gamma^2 L_{max}^2}{\mu^2} \sigma_*^2 + \frac{8\lambda}{\mu} \nonumber  \\
      \iff \exof{\mathcal{E}(x^{k+1}_0, x^*)\given \filter_k} &\leq& c_1 \mathcal{E}(x^k_0, x^*) + c_2
    \end{eqnarray}
    where at the last step we have let $c_1 = 1 - \frac{\gamma n \mu}{2}$ and $c_2 = \frac{\gamma n \mu}{2} + \frac{8 n \gamma^2 L_{max}^2}{\mu^2} \sigma_*^2 + \frac{8\lambda}{\mu}$.
\end{proof}

%\kostas{In Lemma~\ref{lemma: geometric_drift_property} the $x^*$ is any arbitrary $x^* \in\mathcal{X}^*$?}
\begin{lemma}\label{lemma: geometric_drift_property}
   Let Assumptions~\ref{assumt: sol_set_non_empty}-\ref{assumpt: lipschitz} hold. If $\gamma \leq \gamma_{max}$, then for any fixed $x^*\in \mathcal{X}^*$ the functions $\mathcal{E}_1(x, x^*) = \mathcal{E}(x, x^*), \mathcal{E}_2(x, x^*) = \sqrt{\mathcal{E}(x, x^*)}$ satisfy the geometric drift property for the iterates of {Perturbed SGD–\RRresh}, \ie $\forall i \in\{1, 2\}$ there exist measurable set $C_i$, constants $\alpha_i > 0, \tilde{\alpha}_i < \infty$ such that $\forall x\in\R^d$ 
   \begin{equation}
       \Delta \mathcal{E}_i(x, x^*) = - \alpha \mathcal{E}_i(x, x^*) + \one_C \tilde{\alpha}, \label{eq: geometric_drift_property}
   \end{equation}
   where $\Delta \mathcal{E}_i(x, x^*) = \int_{x'\in \R^d} P(z,dx')\energy_i(x') - \energy_i(x)$ and the constant $\gamma_{max} = \gammaub$. 
\end{lemma}
\begin{proof}
In order to prove that the geometric drift property is satisfied, we need to show that there exist function $\energy_i: \R^d\rightarrow [1, +\infty]$, measurable set $C_i$ and constants $\alpha_i>0, \tilde{\alpha_i}<\infty$ such that \eqref{eq: geometric_drift_property} holds. From Corollary~\ref{corol: energy_decrease}, we have that the function $\energy_1: \R^d \rightarrow [1, +\infty]$ with $\energy_1(x, x^*) = \|x - x^*\|^2 + 1$ satisfies along the iterates of {Perturbed SGD–\RRresh} that
\begin{eqnarray}
    \exof{\energy_1(x_{k+1}, x^*) \given \filter_k:\ourbraces{\state_k = x}}&\leq& c_1 \mathcal{E}_1(x, x^*) + c_2 \label{eq: energy_inequality_for_geometric_drift},
\end{eqnarray}
where $c_1 = 1 - \frac{\gamma n \mu}{2}$ and $c_2 = \frac{\gamma n \mu}{2} + \frac{8 n \gamma^2 L_{max}^2}{\mu^2} \sigma_*^2 + \frac{8\lambda}{\mu}$. Additionally, for the epoch-level iterates $x_k$ of~{Perturbed SGD–\RRresh} the definition of $\Delta \energy$ is
\begin{eqnarray}
    \Delta \energy_1(x, x^*) &=& \int_{x' \in \R^d} P(x,\dd x') \energy_1(x', x^*) - \energy_1(x, x^*) \nonumber \\
    &=& \exof{\energy_1(x_{k+1}, x^*) - \energy_1(x_k, x^*) \given \filter_k:\ourbraces{\state_k = x}} \label{eq: Delta_Energy}.
\end{eqnarray}
From \eqref{eq: energy_inequality_for_geometric_drift} and \eqref{eq: Delta_Energy}, we have that
\begin{eqnarray}
    && \exof{\energy_1(x_{k+1}, x^*) \given \filter_k:\ourbraces{\state_k = x}} \leq c_1 \energy_1(x) + c_2 \nonumber\\
    &\Rightarrow& \exof{\energy_1(x_{k+1}, x^*) - \energy_1(x_k, x^*)\given \filter_k:\ourbraces{\state_k = x}}\leq -(1-c_1) \energy_1(x, x^*) + c_2 \nonumber\\
    &\Rightarrow& \Delta \energy_1(x, x^*) \leq -(1-c_1) \energy_1(x, x^*) + c_2 
\end{eqnarray}
Let $C_1 = \left\{x\in\R^d: \energy_1(x, x^*) \leq \frac{2c_2}{(1-c_1)}\right\}$. We have that
\begin{eqnarray}
    \Delta \energy_1(x, x^*) &\leq& -(1-c_1) \energy_1(x, x^*) + \one_C(x) c_2 + \one_{C^{c}}(x) \frac{1-c_1}{2} \energy_1(x, x^*)\nonumber\\
    &\leq& -\frac{1-c_1}{2} \energy_1(x, x^*) + \one_{C_1(x)} c_2 \label{eq: final_delta}
\end{eqnarray}
where at the last step we used the fact that $\one_{C_1^{c}}(x) < 1$ and \textcolor{red}{$c_1 \in (0, 1)$}. 
From \eqref{eq: final_delta} we conclude that $\energy_1(x, x^*)$ satisfies the geometric drift property for the set $C_1 = \left\{x\in\R^d: \energy_1(x) \leq \frac{2c_2}{(1-c_1)}\right\}$ and with constants $\alpha = \frac{1-c_1}{2}, a = c_2$. 

For the $\mathcal{E}_2(x, x^*) = \sqrt{\mathcal{E}(x, x^*)}$, by Jensen's inequality it holds that
\begin{eqnarray}
    \exof{\sqrt{\energy(x_{k+1}, x^*)} \given \filter_k:\ourbraces{\state_k = x}} &\leq& \sqrt{\exof{\energy(x_{k+1}, x^*) \given \filter_k:\ourbraces{\state_k = x}}} \nonumber \\
    &\leq& \sqrt{c_1 \mathcal{E}(x, x^*) + c_2} \nonumber \\
    &\leq& \sqrt{c_1} \sqrt{\energy(x, x^*)} + \sqrt{c_2} \nonumber
\end{eqnarray}
Thus, there exist constants $d_1 = \sqrt{c_1}, d_2 = \sqrt{c_2}$ such that it holds
\begin{eqnarray}
    \exof{\energy_2(x_{k+1}, x^*) \given \filter_k:\ourbraces{\state_k = x}}&\leq& d_1 \mathcal{E}_2(x, x^*) + d_2,
\end{eqnarray}
Since it holds that
\begin{equation}
    \Delta \energy_2(x, x^*) = \int_{x' \in \R^d} P(x,\dd x') \energy_2(x', x^*) - \energy_2(x, x^*)= \exof{\energy_2(x_{k+1}, x^*) - \energy_2(x_k, x^*) \given \filter_k:\ourbraces{\state_k = x}},
\end{equation}
we have that
\begin{eqnarray}
    && \exof{\energy_2(x_{k+1}, x^*) - \energy_2(x_k, x^*)\given \filter_k:\ourbraces{\state_k = x}}\leq -(1-d_1) \energy_2(x, x^*) + d_2 \nonumber\\
    &\Rightarrow& \Delta \energy_2(x, x^*) \leq -(1-d_1) \energy_2(x, x^*) + d_2 
\end{eqnarray}
Let $C_2 = \left\{x\in\R^d: \energy_2(x, x^*) \leq \frac{2d_2}{(1-d_1)}\right\}$. We have that
\begin{eqnarray}
    \Delta \energy_2(x, x^*) &\leq& -(1-d_1) \energy_2(x, x^*) + \one_{C_2(x)} d_2 + \one_{C_2^{c}}(x) \frac{1-d_1}{2} \energy_2(x, x^*)\nonumber\\
    &\leq& -\frac{1-d_1}{2} \energy_2(x, x^*) + \one_{C_2(x)} d_2 \nonumber
\end{eqnarray}
where at the last step we used the fact that $\one_{C_2^{c}}(x) < 1$ and \textcolor{red}{$d_1 \in (0, 1)$}. 
Hence, we conclude that $\energy_2(x, x^*)$ satisfies the geometric drift property for the set $C_2 = \left\{x\in\R^d: \energy_2(x) \leq \frac{2d_2}{(1-d_1)}\right\}$ and with constants $\alpha_2 = \frac{1-d_1}{2}, \tilde{\alpha} = d_2$. 
\end{proof}
%%%%%%%%%%%%%
\subsection{Proof of Minorization Property }\label{app: minorization}
The next step in establishing ergodicity is to verify a \emph{minorization condition}.  
Intuitively, this property guarantees that whenever the chain is in a certain “small set” $C$,  
its one-step transition kernel dominates a fixed nontrivial distribution $\nu$, uniformly with probability $\delta>0$.  
In other words, starting from any $x\in C$, the algorithm has a baseline chance of moving into any region of the state space according to $\nu$.  
This is the key ingredient that, together with a Lyapunov–Foster drift condition, yields geometric ergodicity of the Markov chain.  
The following lemma formalizes this property for the iterates of Perturbed SGD–\RRresh.  
\begin{lemma}[Minorization property]\label{lemma: minorization}
Let Assumptions~\ref{assumt: sol_set_non_empty}--\ref{assumpt: lipschitz} hold.  
If $\gamma \leq \gamma_{\max}$, then the iterates of \texttt{Perturbed SGD--\RRresh} satisfy the minorization condition:  
there exist a constant $\delta>0$, a probability measure $\nu$ on $(\R^d,\mathcal{B}(\R^d))$, and a set $C\subseteq\R^d$ such that $\nu(C)=1$, $\nu(C^\complement)=0$, and
\begin{equation}\label{eq: minorization_cond}
    P(x,A) \;\ge\; \delta\,\one_C(x)\,\nu(A), 
    \qquad \forall x\in\R^d,\; A\in\mathcal{B}(\R^d),
\end{equation}
where $P(x,A)=\Pr(x_{k+1}\in A\mid x_k=x)$ and $\gamma_{\max}=\gammaub$.
\end{lemma}

\begin{proof}
Consider the Lyapunov candidate $\mathcal{E}(x)=\|x-x^*\|^2+1$ for some $x^*\in\mathcal{X}^*$.  
Its sublevel sets 
\[
C(r):=\{x\in\R^d : \mathcal{E}(x)\le r\} = \ball(x^*,\sqrt{r-1}),\qquad r>1,
\]
are bounded, hence suitable for applying small/petite set arguments.  

At each epoch, the update of Perturbed SGD-{\RRresh} can be described by
\[
x_{k+1}=H(x_k,\omega_k)+U_k,
\]
where $\omega_k$ is uniform on $\mathfrak{S}_n$ and $U_k\sim\mathcal N(0,\Sigma I_d)$, independent of $\omega_k$ and $x_k$.  
Thus, for any $A\in\mathcal{B}(\R^d)$,
\[
P(x,A)=\frac{1}{n!}\sum_{\omega\in\mathfrak{S}_n}
\int_A \phi\!\big(y;\,H(x,\omega),\,\Sigma \big)\,dy,
%\frac{1}{n!}\sum_{\omega\in\mathfrak{S}_n}\int_A \phi\!\left((x-y)/\gamma ; H(x,\omega) \,\Sigma I_d\right)\,dy,
\]
Since $\phi(y;m,\Sigma I_d)>0$ for all $y\in\R^d$, the kernel has strictly positive support everywhere.  

Now fix $r_0>1$ and restrict to $C(r_0)$.  
Define the reference measures for any   $A\in\mathcal{B}(\R^d)$


\[\nu(A):=\frac{\operatorname{Leb}(A\cap C(r_0))}{\operatorname{Leb}(C(r_0))}\text{ and }\operatorname{Leb}(A)= 
\int_A   \inf\limits_{x \in C(r_0)} \frac{1}{n!}\sum_{\omega\in\mathfrak{S}_n} \phi\!\big(y;\,H(x,\omega),\,\Sigma \big)\,dy.
\]
i.e., the uniform probability distribution over $C(r_0)$.  
Clearly $\nu(C(r_0))=1$ and $\nu(C(r_0)^\complement)=0$.  

Finally, by continuity of $\phi$ and compactness of $C(r_0)$, there exists $\delta\geq \operatorname{Leb}(C(r_0)) >0$ such that
\[
P(x,A)\;\ge\;\delta\,\nu(A), \qquad \forall x\in C(r_0),\; A\subseteq C(r_0).
\]
If $x\notin C(r_0)$ or $A\not\subseteq C(r_0)$, the right-hand side of \eqref{eq: minorization_cond} is zero and the inequality is trivially satisfied.  
Hence the minorization condition \eqref{eq: minorization_cond} holds.
\end{proof}

%%%%%%%%%%%%%
\subsection{Proof of Irreducibility, Aperiodicity and Harris and Positive Recurrence}
\begin{lemma}\label{lemma: properties-mc}
    The Markov chain $(x_k)_{k \geq 0}$ of {Perturbed SGD–\RRresh} is
    \begin{enumerate}[noitemsep,nolistsep,leftmargin=*]
       \item $\irr-$irreducible for some non-zero $\sigma$-finite measure $\irr$ on $\R^d$ over the Borel $\sigma$-algebra of $\R^d$. 
        \item strongly aperiodic. 
        \item Harris and positive recurrent with an invariant measure.
    \end{enumerate}
\end{lemma}
 \begin{proof}
We prove each of the three properties in turn.  

\paragraph{Irreducibility.}  
From Lemma~\ref{lemma: minorization}, the Markov kernel of \texttt{Perturbed SGD--\RRresh} is
\[
P(x,A)=\frac{1}{n!}\sum_{\omega\in\mathfrak{S}_n}\int_A \phi\!\left(y;\,H(x,\omega),\,\Sigma I_d\right)\,dy,
\qquad A\in\mathcal{B}(\R^d),
\]
where $\phi(\cdot;m,\Sigma I_d)$ is a Gaussian density with strictly positive support.  
Hence, for any measurable set $A$ of positive Lebesgue measure, $P(x,A)>0$.  
Taking $\psi$ to be the Lebesgue measure, we conclude that the chain is $\psi$-irreducible.  

\paragraph{Strong Aperiodicity.}  
By Lemma~\ref{lemma: minorization}, there exist $\delta>0$, a probability measure $\nu$, and a set $C\subseteq\R^d$ such that
\[
P(x,A)\;\ge\;\delta\,\one_C(x)\nu(A), \qquad \forall x\in\R^d,\; A\in\mathcal{B}(\R^d).
\]
Since $C$ has positive Lebesgue measure and $\nu(C) = 1$, $\nu(C^\circ) = 0$ and given that the sets $C(r)$ in the proof of the Lemma \ref{lemma: minorization} are small and of positive measure, we get that the Markov chain is strongly aperiodic.
%
%, this minorization implies strong aperiodicity.  
\paragraph{Harris and Positive Recurrence.}  
By Proposition~\ref{prop: petite_set}, the small set $C$ of Lemma~\ref{lemma: minorization} is also petite.  
Combined with the Foster–Lyapunov drift condition of Lemma~\ref{lemma: geometric_drift_property}, the Geometric Ergodic Theorem (Theorem 15.0.1 in \citep{meyn2012markov}) guarantees that the chain is positive recurrent and admits an invariant probability measure.  

Finally, from Theorem 9.1.8 of \citep{meyn2012markov}, the existence of a Lyapunov function unbounded off petite sets, satisfying $\Delta \mathcal{E} \leq 0$ together with $\psi$-irreducibility, implies Harris recurrence.  
\end{proof}
%%%
%%%\begin{proof}
%%%    We prove each of the three properties: irreducibility, aperiodicity and Harris and positive recurrence, below separately. 
%%%    \begin{itemize}
%%%        \item \textbf{Irreducible.} From the proof of \cref{lemma: minorization} for {Perturbed SGD–\RRresh} we have from \eqref{eq: prob_x_A} that
%%%        \begin{equation*}
%%%            \prob\parens{\state_{\time+1}\in A|\state_\time = x} = \int_{x'\in A} f\left(\dfrac{x-x'}{\gamma}-G_{\omega_k}(x)\right)\dd x',
%%%        \end{equation*}
%%%        where $f$ is the pdf of the normal distribution $\mathcal{N}(0, n^2\gamma^2\sigma_*^2 I)$. 
%%%        
%%%        Let $\irr$ be a non-zero $\sigma$-finite measure in the Borel $\sigma$-algebra of $\R^d$. 
%%%        For any $A\subseteq\borel(\R^d)$ with $\irr(A)>0$, we have that $\ourbraces{x}\subseteq \ball(x,1)$ and there exists $\varepsilon>0$ such that $\ball(a_0,\varepsilon)\subseteq A$, for some $a_0\in A$. Thus, it holds that
%%%        \begin{align*}
%%%            P(x,A) &\geq \int_{\hat{a} \in\ball(a_0,\varepsilon)} f\left(\dfrac{x-\hat{a}}{\step}-G_{\omega_k}(x)\right)\dd \hat{a} \\
%%%            &\geq \int_{\hat{a}\in\ball(a_0,\varepsilon)} \inf_{\hat{x} \in\ball(x,1)} f\left(\dfrac{\hat{x} -\hat{a}}{\step}-G_{\omega_k}(\hat{x})\right)\dd \hat{a} > 0,
%%%        \end{align*}
%%%        where the non-negativity stems from the positive support of the pdf function everywhere in $\R^d$. 
%%%        Thus, we conclude the Markov Chain is $\irr$-irreducible.
%%%        \kostas{After the first inequality, the pdf in Manolis' work was $pdf_{U(\hat{x})}$ where $U(\hat{x})$ is the noise in the stochastic oracle. In our case (finite-sum min problem), how this should be written?}
%%%        
%%%        \item \textbf{Strongly Aperiodic.} 
%%%        From Lemma \ref{lemma: minorization}, we have that there exists constant $\delta > 0$, probability measure $\nu$ and set $C \subseteq \mathbb{R}^d$ such that $\nu(C) = 1$, $\nu(C^\circ) = 0$ and $\forall x \in \mathbb{R}^d$ and any $A \in \mathcal{B}(\mathbb{R}^d)$ it holds 
%%%        \begin{equation*}
%%%            \text{Pr}\left[x_{k+1} \in A | x_k = x\right] \geq \delta \one_{C}(x) \nu(A)
%%%        \end{equation*}
%%%        Given that the sets $C(r)$ in the proof of the Lemma \ref{lemma: minorization} are small and of positive measure, we get that the Markov chain is strongly aperiodic.
%%%
%%%        \item \textbf{Harris \& Positive Recurrent.} By Proposition \ref{prop: petite_set}, the geometric drift property of \ref{lemma: geometric_drift_property} holds for a petite set. According to the Geometric Ergodic Theorem (Theorem 15.0.1) in \citep{meyn2012markov}, since the underlying Markov chain is $\psi$-irreducible and aperiodic we have that it is positive recurrent and has an invariant probability measure. 
%%%
%%%        We, next, show that the chain is Harris recurrent. From the proof of Lemma \ref{lemma: geometric_drift_property} we have that there is a function $\mathcal{E}(x)$ that is unbounded off petite sets satisfying $\Delta \mathcal{E} \leq 0$ and given that the chain is $\psi$-irreducible, we have from Theorem 9.1.8 in \citep{meyn2012markov} that the Markov chain is Harris recurrent. 
%%%\end{itemize}
%%%\end{proof}
%%%


%%%%%%%%%%%%%
\subsection{Proof of Existence of Unique Invariant Distribution at Epoch-level \\(Theorem~\ref{thm: efficient_stats})}
By verifying irreducibility, aperiodicity, and \emph{positive Harris recurrence} \citep{meyn2012markov}, we establish a unique invariant distribution $\pi_\gamma$, geometric convergence in total variation to it, and concentration of scalar observables (admissible test functions) around $x^*$.
\begin{theorem}[Restatement of Theorem~\ref{thm: efficient_stats}] \label{thm: efficient_stats_app}
Under Assumptions~\ref{assumt: sol_set_non_empty}--\ref{assumpt: lipschitz}, run \emph{Perturbed SGD-\RRresh} with $\gamma\le\gamma_{\max}$. Then $(x_k)_{k\ge0}$ admits a unique stationary distribution $\pi_\gamma\in\mathcal P_2(\R^d)$, and additionally:
\begin{align*}
&\text{(i)}~~ |\ex[\ell(x_k)]-\ex_{x\sim\pi_\gamma}[\ell(x)]|\;\le\;c(1-\rho)^k 
&& \forall \ell:|\ell(x)|\le L_\ell(1+\|x\|), \\
&\text{(ii)}~~ |\ex_{x\sim\pi_\gamma}[\ell(x)]-\ell(x^*)|\;\le\;L_\ell\sqrt{C} 
&& \forall \ell:  L_\ell-\text{Lipschitz functions},
\end{align*}
for some $c<\infty$, $\rho\in(0,1)$, $C=\Theta(\mathrm{MSE}(\texttt{SGD}-{\RRresh}))$ and $\gamma_{\max}$ defined in Theorem~\ref{thm: convergence_rate}%$\gamma_{\max}=\gammaub$
\end{theorem}
%%%%%
%%%%%\begin{theorem}[Restatement of Theorem~\ref{thm: efficient_stats}] \label{thm: efficient_stats_app}
%%%%%    Let Assumptions~\ref{assumt: sol_set_non_empty}-\ref{assumpt: lipschitz} hold. If {Perturbed SGD–\RRresh} is run with step size $\gamma < \gammaub$, then the following hold
%%%%%    \begin{enumerate}
%%%%%        \item The epoch-level iterates $(x_k)_{k \geq 0}$ have a unique stationary distribution $\pi_\gamma \in \mathcal{P}_2(\R^d),$ where $P_2(\R^d)$ is the set of distributions supported in $\R^d$ with bounded second moment.
%%%%%        \item For any initialization $x_0 \in \R^d$ and any test function $\ell: \R^d \rightarrow \R$ satisfying 
%%%%%        $\|\ell(x)\| \leq L_{\ell} (1 + \|x\|), \quad \forall x \in \R^d_{\geq 0}$ with $L_{\ell} > 0$, the iterates of {Perturbed SGD–\RRresh} converge geometrically in total variation distance to $\pi_\gamma$, i.e. there exist constants $\rho \in (0, 1)$ and $c \in (0, +\infty)$ such that
%%%%%        \begin{eqnarray}
%%%%%            \left|\ex_{x_k}\left[\ell(x_k)\right] - \ex_{x \sim \pi_\gamma}\left[\ell(x)\right]\right| &\leq& c (1 - \rho)^k
%%%%%        \end{eqnarray}
%%%%%        \item For any $L_{\ell}-$Lipschitz test function $\ell: \R^d \rightarrow \R$, there exists a constant $C \propto \max\{\lambda, \gamma\}/\mu$ such that
%%%%%        \begin{eqnarray}
%%%%%            \left|\ex_{x\sim\pi_\gamma}\left[\ell(x)\right] - \ell(x^*)\right| &\leq& L_{\ell} \sqrt{C}.
%%%%%        \end{eqnarray}
%%%%%    \end{enumerate}
%%%%%\end{theorem}
\begin{proof}
    From Lemma~\ref{lemma: properties-mc}, we have that the underlying Markov Chain has an invariant probability measure. Since from Lemma~\ref{lemma: geometric_drift_property} the induced Markov Chain satisfies the geometric drift property, according to the Strong Ergodic Theorem \citep{meyn2012markov} we conclude that the measure is finite and unique. 
    From the invariant property of $\pi_{\gamma}$, we have that for $x_0 \sim \pi_{\gamma}$ the iterates satisfy also that $(x_k)_{k>0}\sim\pi_{\gamma}$. From Corollary~\ref{corol: energy_decrease}, we have that for an arbitrary fixed $x^*$ the iterates of {Perturbed SGD–\RRresh} with step size $\gamma \leq $ satisfy for $c_1 \in (0, 1), c_2 >0$ that
    \begin{eqnarray}
        \exof{\|x_{k+1} - x^*\|_2^2 + 1 \given \filter_k} &\leq& c_1 \left(\|x_{k} - x^*\|_2^2 + 1\right) + c_2. \nonumber
    \end{eqnarray}
    Taking expectation with respect to the invariant measure $\pi_\gamma$ and using the tower law of expectation, we get
    \begin{eqnarray}
        \ex_{x\sim\pi_{\gamma}}\left[\|x - x^*\|_2^2\right] \leq \frac{c_1 +c_2 -1}{1-c_1} = \mathcal{O}\left(\frac{\max(\gamma, \lambda)}{\mu}\right) < +\infty. \label{eq: big_O_with_max_term}
    \end{eqnarray}
    Combining the above inequality with the fact that $\|x_*\| \leq R$ by Assumption~\ref{assumt: sol_set_non_empty}, we conclude that the invariant measure $\pi_{\gamma}\in \mathcal{P}_2(\R^d),$ where $\mathcal{P}_2(\R^d)$ is the set of distributions supported in $\R^d$ with finite second moment.  

    We, next, proceed with proving the second statement of the Theorem. By assumption, we have that the test function satisfies $\forall x \in \R^d_{\geq 0}$ that 
    \begin{eqnarray}
        |\ell(x)| &\leq& L_{\ell} (1 + \|x\|) \nonumber \\
        &\leq& L_{\ell} (1 + \|x^*\| + \|x - x^*\|) \nonumber \\
        &\leq& L_{\ell} (1 + R + \|x - x^*\|) \nonumber \\
        &\leq& (1 + R)  L_{\ell} (1 + \|x - x^*\|)
    \end{eqnarray}
    where we have used the triangle inequality and the fact that $\|x^*\| \leq R$. Applying Cauchy-Schwarz inequality, we can further upper bound $\|\ell(x)\|$
    \begin{eqnarray}
       |\ell(x)| &\leq& \sqrt{2} (1 + R)  L_{\ell} \sqrt{1 + \|x - x^*\|} \nonumber \\
       &\leq& \max\left(1, \sqrt{2} (1 + R) L_{\ell}\right) \sqrt{\mathcal{E}(x, x^*)} \label{eq: upper_ell_bound}
    \end{eqnarray}
    Letting $c = \max\left(1, \sqrt{2} (1 + R) L_{\ell}\right)$ and $\tilde{\mathcal{E}}(x, x^*) = c \sqrt{\mathcal{E}(x, x^*)}$, we have that
    \begin{eqnarray}
        |\ell(x)| &\leq& \tilde{\mathcal{E}}(x, x^*) \nonumber
    \end{eqnarray}
    From Lemma~\ref{lemma: geometric_drift_property} we have that $\energy_1(x, x^*), \energy_2(x, x^*)$ satisfy the geometric drift property and since $c \geq1$ we have that $\tilde{\energy}(x, x^*) = c \energy_2(x, x^*)$ satisfies also the geometric drift property. According to Theorem 16.0.1 in \citep{meyn2012markov} {Perturbed SGD–\RRresh} is $\tilde{\mathcal{E}}$-uniformly ergodic and there exists $\rho \in (0, 1)$ and $R \in (0, +\infty)$ such that
    \begin{eqnarray}
        \left|P^k \ell(x_0) - \ex_{x\sim\pi_{\gamma}}\left[\ell(x)\right] \right| &\leq& R (1 - \rho)^k \left|\tilde{\mathcal{E}}(x_0, x^*)\right| \label{eq: P_to_be_taken_sup}
    \end{eqnarray}
    Letting $c = R\left|\tilde{\mathcal{E}}(x_0, x^*)\right|$, we have proven the inequality in the statement of the theorem. In order to show that the epoch-level iterates converge under the total variation distance it suffices to consider only functions $\ell: \R^d \rightarrow \R$ that are bounded by 1. In this case, there are constants $\tilde{\rho} \in (0, 1)$ and $\tilde{R} \in (0, +\infty)$ independent of $\ell$ such that it holds 
    \begin{eqnarray}
        \sup_{|\ell| \leq1} \Big|P^k \ell(x_0) - \ex_{x\sim\pi_{\gamma}}\left[\ell(x)\right] \Big| &\leq& \tilde{R} (1 - \tilde{\rho})^k \Big|\tilde{\mathcal{E}}(x_0, x^*)\Big| \nonumber
    \end{eqnarray}
    implying according to the dual representation of Radon metric for bounded initial conditions \citep{wiki} the geometric convergence under the total variation distance. 

    In order to prove the third statement of the theorem, we apply linearity of expectation and the Lipschitz property of the test function $\ell$ and obtain
    \begin{eqnarray}
       \left|\ex_{x\sim\pi_\gamma}\left[\ell(x)\right] - \ell(x^*)\right| &\leq& \ex_{x\sim\pi_\gamma}\left[\left|\ell(x) - \ell(x^*)\right|\right] \nonumber \\
       &\leq& \ex_{x\sim\pi_\gamma}\left[L_{\ell} \left\|x -x^*\right\|\right] \nonumber
    \end{eqnarray}
    Applying Cauchy-Schwarz inequality and using inequality \eqref{eq: big_O_with_max_term}, we obtain that
    \begin{eqnarray}
        \left|\ex_{x\sim\pi_\gamma}\left[\ell(x)\right] - \ell(x^*)\right| \leq L_{\ell} \sqrt{\ex_{x\sim\pi_\gamma}\left[ \left\|x -x^*\right\|\right]} \leq L_{\ell}\sqrt{D} \nonumber
    \end{eqnarray}
    where $D \propto \frac{\max(\gamma, \lambda)}{\mu}$ according to \eqref{eq: big_O_with_max_term}.
\end{proof}
\newpage
We conclude with the establishment of a Law of Large Numbers (LLN) and the corresponding Centra limit Theorem (CLT) that describe the epoch-level iterates of {Perturbed SGD–\RRresh}. 
%%%%%%%%%%%%%
\subsection{Proof of Limit theorems of Epoch-level iterates\\(Theorem~\ref{thm: efficient_stats})}
\begin{theorem}[Restatement of Theorem~\ref{thm: LLN_CLT_res}]
Suppose Assumptions~\ref{assumt: sol_set_non_empty}--\ref{assumpt: lipschitz} hold and run Perturbed SGD--\RRresh with $\gamma \le \gamma_{\max}$, (cf.\ Theorem~\ref{thm: convergence_rate}). \\ 
\\Let $\ell:\R^d\to\R$ be any test function such that $|\ell(x)|\le L_\ell(1+\|x\|^2)$ and $\ex_{x\sim\pi_\gamma}[\ell(x)]<\infty$.\\
Then for the epoch-level iterates, it holds that:
\[
\underbrace{\frac{1}{T}\sum_{t=0}^{T-1}\ell(x_t) 
  \;\xrightarrow{\text{a.s.}}\; \ex_{x\sim\pi_\gamma}[\ell(x)]}_{\text{(LLN)}}
\qquad
\underbrace{T^{-1/2}\sum_{t=0}^{T-1}\!\Big(\ell(x_t)-\ex_{x\sim\pi_\gamma}[\ell(x)]\Big) 
  \;\xrightarrow{d}\; \mathcal N(0,\sigma^2_{\pi_\gamma}(\ell))}_{\text{(CLT)}},
  \]
where $\sigma^2_{\pi_\gamma}(\ell)=\lim_{T\to\infty}\tfrac{1}{T}\ex_{\pi_\gamma}[S_T^2]$ and $S_T^2=\sum_{t=0}^{T-1}\big(\ell(x_t)-\ex_{x\sim\pi_\gamma}[\ell(x)]\big)^2$.
\vspace{-0.25em}
\end{theorem}
%
%\begin{theorem}[Restatement of Theorem~\ref{thm: LLN_CLT_res}]
%    Let Assumptions~\ref{assumt: sol_set_non_empty}-\ref{assumpt: lipschitz} hold. If {Perturbed SGD–\RRresh} is run with step size $\gamma < \gamma_{max}$, then for any function $\ell: \R^d \rightarrow \R$ satisfying $\ex_{x\sim\pi_\gamma}\left[\ell(x)\right] < \infty$ and $|\ell(x)| \leq L_{\ell} (1 + \|x\|^2)$ with $L_{\ell} > 0$ the following hold
%    \begin{enumerate}
%        \item A Law of Large Numbers for the epoch-level iterates of {Perturbed SGD–\RRresh}:
%            \begin{eqnarray*}
%                \lim_{T\rightarrow+\infty} \frac{1}{T}\sum_{t=0}^{T-1} \ell(x_t) = \ex_{x\sim\pi_\gamma}\left[\ell(x)\right] \quad \as
%            \end{eqnarray*}
%        \item A Central Limit Theorem for the epoch-level iterates of {Perturbed SGD–\RRresh}:
%            \begin{eqnarray*}
%                T^{-1/2} \sum_{t=0}^{T-1} \left[\ell(x_t) - \ex_{x\sim\pi_\gamma}\left[\ell(x)\right]\right] \xrightarrow{d} \mathcal{N}(0, \sigma_{\pi_{\gamma}}^2),
%            \end{eqnarray*}
%            where $\sigma_{\pi_{\gamma}}^2(\ell) = \lim\limits_{T\rightarrow+\infty} \frac{1}{T}\ex_{\pi_{\gamma}} \left[S_T^2\right]$ and $S_T^2 = \sum_{t=0}^{T-1} \left[\ell(x_t) - \ex_{x\sim\pi_\gamma}\left[\ell(x)\right]\right]^2$.
%     \end{enumerate}
%     where $\gamma_{max} = \gammaub$. 
%\end{theorem}
\begin{proof}
We show that the Markov Chain induced by the epoch-level iterates of {Perturbed SGD–\RRresh} is Harris positive recurrent, it has an invariant measure and satisfies $\energy$-uniform ergodicity, and hence by Theorem 17.0.1 in \citep{meyn2012markov} the stated Law of Large Numbers and Central Limit Theorem hold. 

From Lemma~\ref{lemma: properties-mc}, we have that the Markov Chain is Harris positive recurrent with an invariant measure. It suffices, thus, to show that the chain is $\energy$-uniform ergodic by proving that there exists a potential function $\energy(\cdot)$ such that the chain satisfies the geometric drift property of \citet{meyn2012markov} and $\|\ell(x)\|^2 \leq \energy(x)$. 
Let $\energy(x, x^*) = \exof{\mathcal{E}(x^{k+1}_0, x^*)\given \filter_k}$ for any fixed $x^*\in\mathcal{X}^*$. According to Lemma~\ref{lemma: geometric_drift_property}, $\energy(x, x^*)$ satisfies the geometric drift property. Additionally, since $\ell$ has a linear growth it holds that
\begin{eqnarray}
    |\ell(x)|^2 &\leq& L_{\ell}^2 (1+\|x\|^2)^2 \nonumber \\
    &\leq& L_{\ell}^2 (1 + \|x^*\| + \|x - x^*\|)^2 \nonumber \\
    &\leq& L_{\ell}^2 (1 + R + \|x - x^*\|)^2 \nonumber \\
    &\leq& L_{\ell}^2 (1 + R)^2  (1 + \|x - x^*\|)^2 \label{eq: upper_bound_ell^2}
\end{eqnarray}
From Cauchy-Schwarz inequality, it holds that
\begin{eqnarray}
    1 + \|x - x^*\| &\leq& \sqrt{2} \sqrt{1+\|x - x^*\|^2} \nonumber \\
    \Rightarrow (1 + \|x - x^*\|)^2 &\leq& 2 (1+\|x - x^*\|^2) \nonumber \\
    \Rightarrow (1 + \|x - x^*\|)^2 &\leq& 2 \energy(x, x^*) \label{eq: up_1}
\end{eqnarray} 
Thus, combining \eqref{eq: up_1} and \eqref{eq: upper_bound_ell^2}, we obtain
\begin{eqnarray}
    |\ell(x)|^2 &\leq& 2 L_{\ell}^2 (1 + R)^2 \energy(x, x^*)
\end{eqnarray}
Thus, $\energy(x, x^*)$ satisfies the geometric drift property and it holds that $|\ell(x)|^2 \leq \energy(x, x^*)$ and hence the chain is $\energy$-uniform ergodic, completing the proof.
\end{proof}
