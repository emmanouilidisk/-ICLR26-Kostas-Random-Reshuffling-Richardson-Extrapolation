\section*{Our model's assumptions.}
While variational inequalities provide a unifying language for optimization, learning, and game dynamics, \emph{no single structural assumption can capture the full complexity of all modern nonconvex--nonconcave problems}. From a computational standpoint, even smooth VIs are intractable in full generality---being tightly connected to Nash equilibria~\citep{
papadimitriou2022computational,goldberg2022ppad}, linear complementarity~\citep{ieor2011linear}, and constrained saddle-point problems~\citep{daskalakis2021complexity}. Consequently, much of the theoretical literature adopts \emph{structured} assumptions (strong convexity, quasi-strong monotonicity, quasar or weak convexity, PL/KL conditions, Minty conditions, error bounds, etc.), each expressive in specific regimes but not universal.

Our work is based on \emph{quasi-strong monotonicity} which falls squarely within this class: it captures stabilizing behaviors of many smooth systems, while remaining far more permissive than strong convexity or global monotonicity. At the same time, it is helpful to clarify that this assumption is not meant as a universal model for all adversarial or fully nonconvex--nonconcave settings. Certain modern ML applications---
including GANs, adversarial robustness, and multi-agent RL---can exhibit fundamentally unstable or rotational dynamics~\citep{Jin2020LocalOptimality,Han2023RiemannianMinimax,KimSeo2022SemiImplicit,Bukharin2023RobustMARL}, where even \emph{local} monotonicity surrogates fail. As such, our theoretical guarantees should be viewed as pertaining to regimes where a minimum amount of local structure is present, rather than to the most adversarial or unstructured cases.
\footnote{A complementary and key fact for our setting, established in \citep[Lemma~A.4]{hsieh2019convergence}, is that \emph{any smooth VI operator is locally quasi-strongly monotone in a neighborhood of a regular solution}. Combined with our Markov-chain recurrence result—which ensures that the iterates remain in such neighborhoods with probability~1—this provides a natural and widely adopted stability regime in which the $\RRresh$ and $\RRrom$ debiasing mechanisms are both theoretically justified and practically meaningful.
}